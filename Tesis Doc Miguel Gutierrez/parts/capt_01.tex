\chapter{INTRODUÇÃO} 
\label{sec:cap1}

\section{CONTEXTUALIZAÇÃO} 
Indivíduos que sofrem lesões ou doenças do sistema neuromuscular deparam-se com défices funcionais temporários e muitas vezes permanentes \cite{Durand2005}. Complicações de saúde provocadas por deficiências decorrentes poderão não apenas afetar o indivíduo afetado, mas também prejudicar a qualidade de vida de sua família e suscitar outros custos socioeconômicos. 

Nesse contexto é possível destacar dois exemplos muito comum no panorama mundial e brasileiro. Primeiramente, a Lesão Medular (\acrshort{LM}), que pode ser causada por múltiplos fatores, encontra-se entre as principais causas de dano neuromuscular, chegando, no Brasil, a uma taxa de incidência aproximada de 11 mil casos a cada ano. Interessante ressaltar que de cada 10 casos, nove envolvem pessoas do sexo masculino \cite{Astur2014, Masini2001}.

A \acrshort{LM} se relaciona com a paralisia e a perda de sensibilidade de várias partes do corpo, provocando atrofia acelerada dos músculos, principalmente os músculos dos membros inferiores \cite{Faria2006}. Contudo, a \acrshort{LM} leva a múltiplas consequências sistêmicas além da atrofia muscular, como o comprometimento de funções autonômicas, que aliado à diminuição de atividades físicas, pode acelerar o surgimento de doenças metabólicas e à piora da função cardiovascular \cite{Faria2006}.

Um segundo cenário de interesse nesse trabalho é a Polineuromiopatia do Doente Crítico (\acrshort{PNMDC}), doença adquirida na Unidade de Tratamento Intensivo (\acrshort{UTI}) \cite{Zamora2013, Miranda2013}. Essa enfermidade é desenvolvida na \acrshort{UTI} devido a desordens neuromusculares que causam fraqueza generalizada e dependência prolongada de ventilação mecânica. A \acrshort{PNMDC} é relacionada também com a disfunção, não só do sistema neuromuscular, mas também de múltiplos órgãos e sistemas, o que promove uma permanência prolongada na \acrshort{UTI} e redução gradativa da probabilidade de sobrevivência.

Com o aperfeiçoamento continuado de novas tecnologias para tratamento e reabilitação na \acrshort{UTI}, o Doente Critico (\acrshort{DC}) é então mantido por um período prolongado até seu óbito ou alta hospitalar.  Geralmente no Brasil em média 69\% dos pacientes que são internados na \acrshort{UTI} com diagnóstico de pronta recuperação podem permanecer de 1 a 7 dias, em contrapartida, aproximadamente 31\% dos doentes críticos ultrapassa os sete dias \cite{Abelha2006} \cite{Abelha2006}. Nesse período, então, além do motivo de internação, o \acrshort{DC} entra em um estado permanente de imobilização, levando-o principalmente a adicionar a seu quadro clínico, o desuso muscular e consequentemente ao desenvolvimento de fraqueza muscular generalizada \cite{Miranda2013}.

Uma possível forma de limitar os efeitos das disfunções neuromusculares características da \acrshort{LM} e da \acrshort{PNMDC} reside no fato de que grande parte dos casos o paciente apresenta nervos periféricos intactos, possibilitando a aplicação de terapias que auxiliam na ativação de tais funções neuromusculares. Para isto, técnicas de reabilitação têm sido desenvolvidas no contexto da fisioterapia, como a eletroterapia \cite{Grill2000, McDonald2002}. 

A fisioterapia é uma ciência da saúde que estuda, previne e trata os distúrbios cinéticos funcionais intercorrentes em órgãos e sistemas do corpo humano, gerados por alterações genéticas, por traumas e por doenças adquiridas. Em ambientes como a \acrshort{UTI}, por exemplo, o fisioterapeuta trata o paciente crítico para manter e prevenir vários aspectos da fisiologia em virtude da dependência total ou parcial dos pacientes que podem culminar na chamada Síndrome do Imobilismo ou até a \acrshort{PNMDC} \cite{Franca2012}. 

A eletroterapia é uma técnica que abrange a medicina física e a reabilitação. É definida como a ciência do tratamento de lesões e doenças neuromusculares por meio de estímulos elétricos \cite{Crepon2008}. Nesse trabalho, estamos especialmente interessados na aplicação de estímulos elétricos que irão induzir a despolarização do nervo periférico para produção de atividade neuromuscular como uma alternativa ao exercício ativo. A técnica de eletroestimulação (em inglês electrical stimulation, \acrshort{ES}) caracteriza-se por corrente elétrica de baixa energia aplicada no conjunto pele-nervo-músculos em geral por meio de eletrodos superficiais com a finalidade de evocar uma contração artificial de músculos esqueléticos \cite{Popovic2000}.

Considerando as condições supracitadas, pesquisas mostram que a estimulação elétrica pode ser usada como alternativa ao exercício ativo e mobilização de pacientes acamados como o paciente com \acrshort{PNMDC} e pessoas com \acrshort{LM}. Nesses casos, o uso da estimulação elétrica tem demostrado sua vantagem em comparação a outras terapias em termos de prevenção, recuperação e melhora das condições de saúde dos pacientes com funções neuromusculares comprometidas ou perdidas \cite{Faria2006, Silva2016,Deley2005,Zanotti2003, Naki2011}. De fato, a atividade neuromuscular provocada pela estimulação elétrica se traduz em uma melhoria não só deste sistema, mas também de outros sistemas, como o circulatório e pulmonar \cite{Ferreira2014}. 

Tradicionalmente, existem vários termos relacionados à aplicação da eletroestimulação, como Estimulação Elétrica Funcional (\acrshort{EEF}) ou \acrshort{FES} (do inglês \textit{Functional Electrical Stimulation}), a Estimulação Elétrica Neuromuscular (\acrshort{EENM}) ou \acrshort{NMES} (do inglês \textit{Neuromuscular Electrical Stimulation}), a Estimulação Elétrica Nervosa Transcutânea (\acrshort{EENT}) ou \acrshort{TENS} (do inglês \textit{Transcutaneous Electrical Nerve Stimulation}), entre outros. Na fisioterapia, o termo utilizado muitas vezes refere-se ao tipo de aplicação da estimulação elétrica, ou seja, da terapia prescrita. Consequentemente, cada terapia possuirá diferentes parâmetros de estimulação, como forma de onda, largura do pulso, frequência.

Neste trabalho, não estabelecemos preferência acerca do termo utilizado para descrever a aplicação de estímulo elétrico para indução de atividade neuromuscular. Por outro lado, pretende-se investigar de maneira detalhada métodos para parametrização e produção de estímulos elétricos para aplicação em diferentes tipos de exercícios assistidos por estimulação elétrica. Entre os diferentes tipos de exercícios utilizados na prática clínica, como exercícios isométricos, resistivos (com uso de pesos e/ou elásticos) e acoplados em cicloergômetros (como ciclo-ergômetro, remo-ergômetro).

\section{DEFINIÇÃO DO PROBLEMA}

Muito embora estudos já tenham demonstrado os benefícios do uso da eletroestimulação para reduzir efeitos decorrentes da \acrshort{PNMDC} e da diminuição da atividades neuromuscular em pessoas com \acrshort{LM}, atualmente pesquisadores ainda procuram estabelecer quais são os parâmetros de estimulação (\acrshort{PE}) ótimos para diferentes indivíduos e tipos de exercício \cite{Silva2016}. Em outras palavras, ainda falta estabelecer um sistema padrão que identifique os \acrshort{PE} adequados ao tipo de terapia e ao aparecimento/retraso da fadiga.

Em fisioterapia, uma vez prescrita a terapia para fortalecimento muscular, a definição dos \acrshort{PE} inicia com o posicionamento dos eletrodos. A depender do protocolo, o posicionamento se dá sobre nervos periféricos que geram ativação dos músculos de interesse ou sobre o próprio ventre muscular \cite{Kitchen2003a}. A verificação visual da contração é muitas vezes usada para validar essa etapa. Logo depois do posicionamento dos eletrodos, devem ser selecionados os PE, incluindo parâmetros de baixo nível (como largura de pulso, frequência, amplitude) e alto nível (duração da terapia, razão entre período de repouso e atividade, entre outros). A definição de ambos os conjuntos de parâmetros é realizada usualmente com base em orientações estabelecidas na literatura para a população em geral e, particularmente para os parâmetros de baixo nível, em conjunto com a verificação empírica do nível de contração obtido. 

Um dos problemas apresentados por essa metodologia é a subjetividade inerente da detecção visual da contração e, principalmente, a necessidade de contar com dedicação completa do terapeuta caso deseje-se monitorar o nível de contração muscular obtido durante a terapia e a compensação da fadiga decorrente da mesma. Nesse contexto, uma das alternativas é o exame identificado na literatura como Eletrodiagnóstico de Estímulo (\acrshort{EDE}) que conta com três tipos de teste Reobase, cronaxia e acomodação (ver \ref{cap:sec3.2.6}) os quais permite estimar algumas propriedades relativas à excitabilidade neuromuscular \cite{Kimura2013}. Muito embora o \acrshort{EDE} seja utilizado primordialmente para diagnóstico de doenças neuromusculares, recentemente foi proposto um protocolo de eletroestimulação baseado na cronaxia para ajudar a parametrizar a eletroterapia \cite{Silva2016}.  

Dentre os testes do \acrshort{EDE} ressalta a Reobase
com o qual seria possível determinar parâmetros padronizados de excitabilidade neuromuscular, em especial a amplitude. Ainda nesse contexto o exame requer a participação direta do terapeuta para identificar a ocorrência de uma contração muscular claramente visível\footnote{Neste trabalho usa-se os termos Excitabilidade Neuromuscular \acrshort{ExN} para referir se a quantidade de corrente que gera uma contração muscular claramente visível e detectável por meio de sensores de movimento}. Com o intuito de otimizar a produtividade do profissional nessa tarefa, neste trabalho propõe-se o uso de sensores de movimento para identificar a contração muscular de forma automática. Assim, em termos práticos, um protocolo de fortalecimento isométrico poderá ser personalizado de forma automática, sobretudo em termos de parâmetros de baixo nível e detecção/compensação de fadiga.

Caso de fato o teste de \acrshort{ExN} possa ser utilizado para estimar e monitorar o nível da atividade muscular em exercícios isométricos, uma pergunta adicional seria se métodos similares poderiam ser aplicados em exercícios dinâmicos. Nesse caso, uma alternativa para exercícios com carga envolveria a própria medição do movimento articular realizado, por exemplo a flexão plantar de tornozelo ou extensão de joelho. Neste caso, por exemplo, a redução do nível de contração muscular poderia ser detectada, por exemplo, pela diminuição do ângulo de extensão do joelho.

Uma justificativa adicional para inclusão da medição da excitabilidade muscular e/ou medição do movimento em exercícios assistidos por \acrshort{ES} é a possibilidade de monitorar a fadiga muscular sem a necessidade de incorporar instrumentos adicionais (como sensores de força ou eletromiografia) \cite{Vollestad1997}. Na pratica a estimação da fadiga pode ser utilizada para modular a intensidade de ES e assim permitir a manutenção do nível de contração muscular inicialmente estabelecido pelo terapeuta. De fato, em muitos protocolos de fortalecimento baseados na contração voluntária os exercícios são realizados em termos de repetições com a carga constante, i.e. com correção da intensidade de contração voluntariamente em cada repetição. Entretanto atualmente o fortalecimento assistido por eletroestimulação se dá em malha aberta, i.e. sem a correção do nível de contração a partir do surgimento da fadiga.

Por fim, o estabelecimento de um protocolo efetivo de exercício assistido por \acrshort{ES} requer que os \acrshort{PE} sejam efetivamente produzidos pelo equipamento. Por certo, no mercado de equipamentos médicos existem diversos tipos de eletroestimuladores, sendo indicados para múltiplos usos em áreas clínicas e esportivas. Entretanto, além de não permitirem a realização do eletrodiagnóstico e monitoramento de forma integrada, existe em alguns casos irregularidade nos parâmetros de ES gerados, sobretudo para condições de maior carga e em indivíduos com a excitabilidade neuromuscular alterada, como pessoas com \acrshort{LM} e \acrshort{PNMDC}. 

Assim, neste trabalho pretendesse verificar a hipótese de si é possível construir um novo sistema de \acrshort{ES} que integre elementos de medição de movimento induzido visando identificar a fadiga em exercícios isométricos ou resistidos por meio da medição da excitabilidade neuromuscular para terapia e treinamento em pessoas com \acrshort{LM} e/ou \acrshort{PNMDC}.

Assim, nesse trabalho o projeto de um novo sistema de \acrshort{ES} que integra elementos de medição do movimento induzido em combinação com o desenvolvimento de novo circuito de \acrshort{ES}. 

\section{OBJETIVOS}

\subsection{PRINCIPAL}
Projetar, implementar e testar um novo sistema eletrônico de estimulação elétrica que permita controlar os parâmetros de estimulação em uma grande faixa de operação, assim como integrar um subsistema de detecção automático de fadiga baseado na medição da excitabilidade neuromuscular e ajuste em tempo real do protocolo de terapia com o intuito de minimizar os efeitos de lesões ou doenças no sistema neuromuscular de indivíduos com lesão medular e/ou pacientes com polineuromiopatia.  

\subsection{ESPECÍFICOS}
Os objetivos específicos deste trabalho são:
\begin{itemize}
\item Desenvolver novo equipamento de estimulação elétrica multicanal e validar o sistema com base não apenas nas normas brasileiras geral e particular de equipamentos eletromédicos para eletroestimulação, mas também em critérios de precisão do sinal de eletroestimulação gerado;
\item Implementar uma interface de usuário amigável baseada em PC para controle do novo equipamento eletromédico; 
\item Integrar ao equipamento um subsistema de medição de movimento induzido por ES baseado em sensores inerciais;
\item Implementar um subsistema para realização automatizada da medição de excitabilidade neuromuscular e validar tal subsistema em testes de bancada;
\item Implementar sistema de estimação do movimento articular induzido por ES baseado em sensores inerciais para exercícios isométricos e resistidos de um grau de liberdade;
\item Verificar a hipótese de que o teste de reobase pode ser usado como um indicativo da fadiga quando se aplica a estimulação elétrica em procedimentos de terapia e/ou treinamento com exercícios isométricos ou resistidos.
\item Estimar a fadiga induzida por \acrshort{ES} usando o \acrshort{EDE} automatizado e a estimação do movimento;
\item Implementar sistema de controle para regulação da contração muscular a partir da estimação da fadiga;
\item Realizar validação clínica preliminar do sistema desenvolvido em voluntários sem excitabilidade neuromuscular alterada e em indivíduos com excitabilidade neuromuscular alterada.
\end{itemize}

\section{ORGANIZAÇÃO DO MANUSCRITO}
No capítulo 2 é apresentada uma revisão dos fundamentos teóricos relacionados com o sistema neuromuscular, a eletroestimulação e os dispositivos de eletroestimulação. Em seguida, o capítulo 3 descreve o estado da arte do tema investigado, incluindo perspectiva histórica, sistemas contemporâneos de eletroestimulação e sistemas de estimulação disponíveis comercialmente. O capítulo 4 apresenta o desenvolvimento do sistema proposto, incluindo desde aspectos relativos aos requisitos de hardware e software até simulações numéricas de circuitos propostos. Os resultados obtidos até o momento e análise dos mesmos são apresentados no capítulo 5. Finalmente, o capítulo 6 expões as considerações dessa etapa do trabalho e descreve a proposta de trabalho para a conclusão dessa tese de doutorado.